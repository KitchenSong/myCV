%based on the guide at http://www.cv-templates.info/2009/03/professional-cv-latex/
\documentclass[a4paper,10pt]{article}

%A Few Useful Packages
\usepackage{marvosym}
\usepackage{fontspec} 					%for loading fonts
\usepackage{xunicode,xltxtra,url,parskip} 	%other packages for formatting
\RequirePackage{color,graphicx}
\usepackage[usenames,dvipsnames]{xcolor}
\usepackage[big]{layaureo} 				%better formatting of the A4 page
% an alternative to Layaureo can be ** \usepackage{fullpage} **
\usepackage{supertabular} 				%for Grades
\usepackage{titlesec}					%custom \section
\usepackage{paralist}
\usepackage{geometry}                     %for setting the margin
\usepackage{multirow}
\usepackage{array}
\usepackage{xcolor}
%Setup hyperref package, and colours for links
\usepackage{hyperref}
\usepackage{marvosym} %symbol
\definecolor{linkcolour}{rgb}{0,0.2,0.6}
\hypersetup{colorlinks,breaklinks,urlcolor=linkcolour, linkcolor=linkcolour}

%FONTS
\defaultfontfeatures{Mapping=tex-text}
\setmainfont[ Path = fonts/, BoldFont = Fontin-Bold, BoldItalicFont = Fontin-Bold, ItalicFont = Fontin-Italic, SmallCapsFont = Fontin-SmallCaps]{Fontin-Regular}%SmallCaps]{Fontin-Regular}
%\setmainfont[Path = fonts/, BoldFont = LinLibertine_RB.otf, ItalicFont = LinLibertine_RI.otf, BoldItalicFont = LinLibertine_RBI.otf, SlantedFont = LinLibertine_aRL.otf, BoldSlantedFont = LinLibertine_aBL.otf, SmallCapsFont = LinLibertine_aS.otf]{LinLibertine_R.otf}
%\setmainfont[ Path = fonts/, BoldFont = GILB, BoldItalicFont = GILBI, ItalicFont = GILI, SmallCapsFont = GIL]{GIL}

%CV Sections inspired by:
%http://stefano.italians.nl/archives/26
\titleformat{\section}{\Large\scshape\raggedright}{}{0em}{}[\titlerule]%\scshape\raggedright}{}{0em}{}[\titlerule]
\titlespacing{\section}{0pt}{3pt}{3pt}
%Tweak a bit the top margin
%\addtolength{\voffset}{-1.3cm}

%Italian hyphenation for the word: ''corporations''
\hyphenation{im-pre-se}
\newgeometry{left=1.5cm,right=1.5cm,top=1.7cm, bottom=0.5cm}  %20140829 top 1.5-->2.5
%--------------------BEGIN DOCUMENT----------------------
\begin{document}

\pagestyle{empty} % non-numbered pages

\font\fb="[cmr10]" %for use with \LaTeX command
\centering
  {\Huge \bf{\textsc{Qichen} \textsc{\textbf {Song}}}%\textbf{{\Huge{Q}\Large{ICHEN} \Huge{S}\Large{ONG}}%
\vspace{-0.2em}}\bigskip

%-------------------Demarcation---------
\hrule height 0.7mm \vspace{-0.4em}
\begin{tabular}{p{9.8cm}p{3.1cm}p{4.0cm}}
%------------------Basic info-----------
 %  \textsc{Address:}
   {\small Zisong6\#325, Huazhong Univ. of Sci. \& Tech., Wuhan, 430074, China}
  %&\textsc{Phone:}
  & \large{\Telefon} {\small +86 131 6323 8726}
   %\multirow{3}{*}{ \Huge Qichen \textsc{Song}} \\
  % &\textsc{email:}
   & \large{\Letter} {\small \href{mailto:kitchensong@gmail.com}{\textcolor{black}{kitchensong@gmail.com}}}\\
%--------------------TITLE-------------
%\par{
 % { \Huge Qichen \textsc{Song}
%}\bigskip\par}
\end{tabular}
%--------------------SECTIONS-----------------------------------

%Section: Education

\section{Education}

\begin{tabular}{rp{11.7cm}|l}
    &  \hspace{-1em} \textbf{Huazhong University of Science and Technology (HUST)}, \emph{2011.09-present}&\textbf{Standard Tests}\\
   &\hspace{-1em} Major: Thermal Energy and Power Engineering
         &TOFEL: 107 (R29 L30 S23 W25) \\
&\hspace{-1em} Degree: Bachelor of Engineering, expected \emph{2015.06} &GRE: V152+Q170+AW4.0 \\
&\hspace{-1em} Overall GPA: \textbf{92.2/100} \hspace{1em} Overall Rank: \textbf{1/366}&\\
\end{tabular}

%Section: Research Experience
\section{Research Experience}
\begin{tabular}{p{13.5cm}p{0.5cm}r}
\textbf{Research on coupling between different phonon modes in graphene} && \emph{2014.09-present} \\
\hspace{1em} Advisor: Prof. Nuo Yang, Dr. Meng An \hspace{3em} \emph{Nano Heat Group} && \vspace{-0.5em}\\
\begin{compactitem}
       %\item Independly programming \vspace{0.2em}
       \item Manipulating in-plane and out-of-plane temperature gradient separately\vspace{0.2em}
       \item Investigating coupling between different phonon modes (TA, LA and ZA) and their contributions to thermal conductivity\vspace{0.2em}
     \end{compactitem}&&\vspace{-2.2em} \\
\multicolumn{3}{c}{} \\
\textbf{Research on modulation of thermal conductivity in folded graphene} && \emph{2013.11-2014.09} \\
\hspace{1em} Advisor: Prof. Nuo Yang \hspace{9em} \emph{Nano Heat Group} && \vspace{-0.5em}\\
\begin{compactitem}
       \item Independently wrote code of nonequilibrium molecular dynamics (NEMD)\vspace{0.2em}
       \item Designed innovative structure to reduce the thermal conductivity significantly\vspace{0.2em}
       \item Obtained size-independent thermal conductivity that characterizes large-sized folded graphene's thermal properties
     \end{compactitem}&&\vspace{-2.2em} \\
\multicolumn{3}{c}{} \\
\textbf{Research on the temperature and flow field analysis of sapphire crystal growth}  && \emph{2013.08-2013.11} \\
\hspace{1em} Advisor: Prof. Haisheng Fang \hspace{6.7em} \emph{Multiscale Process Modeling Lab}  && \vspace{-0.5em} \\
\begin{compactitem}
       \item Comprehensively investigated varied flow fields' influence on sapphire growth\vspace{0.2em}
       \item Used Discrete Phase Model to investigate the distribution of inert impurities\vspace{0.2em}
       \item Simplified the complex system and found a new way to improve sapphire's quality %Investigated the relationship between the quality of the sapphire and the rotation speed
     \end{compactitem}&&\vspace{-2.2em} \\
\multicolumn{3}{c}{} \\
 \textbf{Team leader on designing the device utilizing wave energy in small watersheds}&&  \emph{2013.05-2013.08} \\
\hspace{1em}  Advisor: Prof. Jun Xiang  & &\vspace{-0.5em}\\
\begin{compactitem}
       \item Designed the innovative machine to collect and convert the wave energy\vspace{0.2em}
       \item Successfully optimized the structure by modeling and effectively improved the conversion efficiency\vspace{0.2em}
       \item Made the prototype of the device
     \end{compactitem}&&\vspace{-1em} \\
\multicolumn{3}{c}{} \vspace{-1.5em} \\
  %\textsc{2007 - 2008} & \emph{Intern developer} at \textbf{PT Comunicações}, Lisboa\\
  %& \begin{compactitem}
  %  \item Helped to develop core funcionality in an IT infrastructure monitoring solution
 %  \end{compactitem}\vspace{-2em} \\
%\multicolumn{2}{c}{} \\


\end{tabular}

\section{Patent}
\begin{tabular}{p{15.8cm}l}
\textbf{Q.C. Song} et al, ”An electricity generating device by utilizing small wave energy” (submitted \emph{2014}) &\\
\end{tabular}

\section{Honors and Awards}
\begin{tabular}{p{14.5cm}r}
\textbf{National Scholarship} (Three times) & \emph{2012,2013,2014}  \\
 \hspace{1em} {\small Top 1\% among all competitors, awarded by Ministry of Education of PRC}& \vspace{0.2em} \\
\textbf{Outstanding Student of Huazhong Univ. of Sci. $\&$ Tech.}  & \emph{2012-2014}\\
 \hspace{1em} {\small Top 1\% among all 2nd \& 3rd year students, one of the top honor for undergraduates}& \vspace{0.2em} \\
\textbf{Merit Student} (Three times) &  \emph{2012,2013,2014}  \\
 \hspace{1em} {\small Top 4\% among all competitors, issued by HUST}&  \vspace{0.2em}\\
\textbf{Excellent Award in the 3rd National Water Resource Innovation Design Competition}  & \emph{2013.07}\\
\end{tabular}


\section{Internship Experience}
\begin{tabular}{p{15.5cm}r}
\textbf{Summer Intership at Shangu Power Co.,Ltd.}, Xi'an &\emph{2014.06}\vspace{-0.5em} \\
\begin{compactitem}
       \item  Learned details of manufacturing process of tail gas turbine
       \item  Systematically learned the CFD calculation methods for turbine design
     \end{compactitem}&\vspace{-2em} \\
\multicolumn{2}{c}{}\vspace{-0.5em} \\
\end{tabular}
\section{Computer Skills}
\begin{tabular}{p{15.8cm}l}
 \hspace{-1em} FORTRAN90(MPI), C++, Fluent, AutoCAD, MATLAB/Simulink, {\fb \LaTeX}  &\\
\end{tabular}
\end{document}
