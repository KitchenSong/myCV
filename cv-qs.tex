%based on the guide at http://www.cv-templates.info/2009/03/professional-cv-latex/
\documentclass[a4paper,10pt]{article}

%A Few Useful Packages
\usepackage{marvosym}
\usepackage{fontspec} 					%for loading fonts
\usepackage{xunicode,xltxtra,url,parskip} 	%other packages for formatting
\RequirePackage{color,graphicx}
\usepackage[usenames,dvipsnames]{xcolor}
\usepackage[big]{layaureo} 				%better formatting of the A4 page
% an alternative to Layaureo can be ** \usepackage{fullpage} **
\usepackage{supertabular} 				%for Grades
\usepackage{titlesec}					%custom \section
\usepackage{paralist}
\usepackage{geometry}                     %for setting the margin
\usepackage{multirow}
\usepackage{array}
%Setup hyperref package, and colours for links
\usepackage{hyperref}
\usepackage{marvosym} %symbol
\definecolor{linkcolour}{rgb}{0,0.2,0.6}
\hypersetup{colorlinks,breaklinks,urlcolor=linkcolour, linkcolor=linkcolour}

%FONTS
\defaultfontfeatures{Mapping=tex-text}
\setmainfont[ Path = fonts/, BoldFont = Fontin-Bold, BoldItalicFont = Fontin-Bold, ItalicFont = Fontin-Italic, SmallCapsFont = Fontin-SmallCaps]{Fontin-Regular}
%\setmainfont[Path = fonts/, BoldFont = LinLibertine_RB.otf, ItalicFont = LinLibertine_RI.otf, BoldItalicFont = LinLibertine_RBI.otf, SlantedFont = LinLibertine_aRL.otf, BoldSlantedFont = LinLibertine_aBL.otf, SmallCapsFont = LinLibertine_aS.otf, Mapping = tex-text]{LinLibertine_R.otf}

%CV Sections inspired by:
%http://stefano.italians.nl/archives/26
\titleformat{\section}{\Large\scshape\raggedright}{}{0em}{}[\titlerule]
\titlespacing{\section}{0pt}{3pt}{3pt}
%Tweak a bit the top margin
%\addtolength{\voffset}{-1.3cm}

%Italian hyphenation for the word: ''corporations''
\hyphenation{im-pre-se}
\newgeometry{left=1.5cm,right=1.5cm,top=2.0cm, bottom=0.5cm}  %20140829 top 1.5-->2.5
%--------------------BEGIN DOCUMENT----------------------
\begin{document}

\pagestyle{empty} % non-numbered pages

\font\fb="[cmr10]" %for use with \LaTeX command
\centering
  { \Huge Qichen \textsc{Song}
\vspace{0.7em}}\bigskip

%-------------------Demarcation---------
\hrule height 0.7mm \vspace{-0.4em}
\begin{tabular}{p{10cm}p{2.9cm}p{4.9cm}}
%------------------Basic info-----------
 %  \textsc{Address:}
   {\small ZISONG6\#325, Huazhong Univ. of Sci. \& Tech., Wuhan, 430074, China}
  %&\textsc{Phone:}
  & \Telefon \hspace{0.1em} {\small +86 13163238726}
   %\multirow{3}{*}{ \Huge Qichen \textsc{Song}} \\
  % &\textsc{email:}
   & \Letter \hspace{0.1em} {\small \href{mailto:kitchensong@gmail.com}{kitchensong@gmail.com}}\\
%--------------------TITLE-------------
%\par{
 % { \Huge Qichen \textsc{Song}
%}\bigskip\par}
\end{tabular}
%--------------------SECTIONS-----------------------------------

%Section: Education

\section{Education}

\begin{tabular}{rp{12cm}|r}
    &  \hspace{-1em} \textbf{Huazhong University of Science and Technology (HUST)}\textsc{, Sep. 2011 -} \textsc{Present}&\textbf{Standard Tests}\\
   &\hspace{-1em} \textsc{Major:} Thermal Energy and Power Engineering
         &GRE: V152+Q170+AW4 \\
&\hspace{-1em} \textsc{Degree:} Bachelor of Engineering, expected June 2015 &TOFEL: 107 (R29 L30 S23 W25)\\
&\hspace{-1em} \textsc{Overall GPA:} \textbf{92.6/100} \hspace{1em} \textsc{Overall Rank:} \textbf{1/366}&\\
\end{tabular}

%Section: Research Experience
\section{Research Experience}
\begin{tabular}{p{14cm}r}
\textbf{Research on the thermal conductivity of folded graphene}  &\textsc{Nov. 2013 - present} \\
\hspace{1em}\textsc{Supervisor:} \emph{Prof. Nuo Yang \hspace{19em}} \emph{Nano Heat Group} & \vspace{-0.5em}\\
\begin{compactitem}
       \item Simulating the evolution process by nonequilibrium molecular dynamics (NEMD)
       \item Designing innovative structure to reduce the thermal conductivity
       \item Modifying the parameters of structure to obtain a converged outcome
     \end{compactitem}&\vspace{-1em} \\
\multicolumn{2}{c}{} \\
\textbf{Research on the thermal and fluid field analysis of sapphire crystal growth}  &\textsc{Nov. 2013 - May 2014} \\
\hspace{1em} \textsc{Supervisor:} \emph{Prof. Haisheng Fang} \hspace{10em} \emph{Multiscale Process Modeling Lab}  & \vspace{-0.5em} \\
\begin{compactitem}
       \item Analyzed the velocity field by using Computational Fluid Dynamics software
       \item Used Discrete Phase Model to investigate the distribution of inert impurities
       \item Investigated the relationship between the quality of the sapphire and the rotation speed
     \end{compactitem}&\vspace{-1em} \\
\multicolumn{2}{c}{} \\
 \textbf{Team leader on designing the device utilizing wave energy in small watersheds}  &\textsc{May 2013 - Aug. 2013} \\
\hspace{1em}  \textsc{Supervisor:} \emph{Prof. Jun Xiang}  & \vspace{-0.5em}\\
\begin{compactitem}
       \item Designed and optimized the shape of the floating part
       \item Designed the core component to collect and convert the wave energy
       \item Made the prototype of the device
     \end{compactitem}&\vspace{-1em} \\
\multicolumn{2}{c}{} \vspace{-1.5em} \\
  %\textsc{2007 - 2008} & \emph{Intern developer} at \textbf{PT Comunicações}, Lisboa\\
  %& \begin{compactitem}
  %  \item Helped to develop core funcionality in an IT infrastructure monitoring solution
 %  \end{compactitem}\vspace{-2em} \\
%\multicolumn{2}{c}{} \\


\end{tabular}

\section{Patent}
\begin{tabular}{ll}
\textbf{Q.S. Song} et al, ”An electricity generating device by utilizing small wave energy” (patent submitted 2014) &\\
\end{tabular}

\section{Honors and Awards }
\begin{tabular}{p{14cm}r}
\textbf{National Scholarship} (Three times) & \textsc{2012 $\&$ 2013 $\&$ 2014}  \\
 \hspace{1em} {\small Top 1\% among all competitors, awarded by Ministry of Education of PRC.}&  \\
\textbf{Outstanding Student of Huazhong University of Sci. $\&$ Tech.} (Three times) & \textsc{2012 $\&$ 2013 $\&$ 2014}\\
 \hspace{1em} {\small Top 1\% among all 2nd \& 3rd year students, one of the top honor for undergraduates.}& \\
\textbf{Merit Student} (Three times) & \textsc{ 2012 $\&$ 2013 $\&$ 2014}  \\
 \hspace{1em} {\small Top 4\% among all competitors, issued by HUST.}&  \\
\textbf{Excellent Award in the 3rd National Water Resource Innovation Design Competition}  & \textsc{July 2013}\\
\end{tabular}


\section{Internship Experience}
\begin{tabular}{p{15.7cm}r}
\textbf{Summer Intership at Shangu Power Co.,Ltd.}, Xi'an &\textsc{July 2014 }\vspace{-0.5em} \\
\begin{compactitem}
       \item  Learned the manufacturing process of axial compressor
       \item  Learned the experimental method of rotator moving equilibrium
       \item  Learned the CFD calculation of compressor and turbine design
     \end{compactitem}&\vspace{-2em} \\
\multicolumn{2}{c}{}\vspace{-0.5em} \\
\end{tabular}
\section{Computer Skills}
\begin{tabular}{p{16cm}r}
 \hspace{-1em} FORTRAN90 (MPI), C++, Fluent, AutoCAD, MATLAB/Simulink, {\fb \LaTeX}  &\\
\end{tabular}
\end{document}
