%based on the guide at http://www.cv-templates.info/2009/03/professional-cv-latex/
\documentclass[a4paper,12pt]{article}

%A Few Useful Packages
\usepackage{marvosym}
\usepackage{fontspec} 					%for loading fonts
\usepackage{xunicode,xltxtra,url,parskip} 	%other packages for formatting
\usepackage{setspace} %spacing
\RequirePackage{color,graphicx}
\usepackage[usenames,dvipsnames]{xcolor}
\usepackage[big]{layaureo} 				%better formatting of the A4 page
% an alternative to Layaureo can be ** \usepackage{fullpage} **
\usepackage{supertabular} 				%for Grades
\usepackage{titlesec}					%custom \section
\usepackage{paralist}
\usepackage{geometry}                     %for setting the margin
\usepackage{multirow}
\usepackage{array}
\usepackage{xcolor}
%Setup hyperref package, and colours for links
\usepackage{hyperref}
\usepackage{marvosym} %symbol
\definecolor{linkcolour}{rgb}{0,0.2,0.6}
\hypersetup{colorlinks,breaklinks,urlcolor=linkcolour, linkcolor=linkcolour}

%FONTS
\defaultfontfeatures{Mapping=tex-text}
\setmainfont[ Path = fonts/, BoldFont = Fontin-Bold, BoldItalicFont = Fontin-Bold, ItalicFont = Fontin-Italic, SmallCapsFont = Fontin-SmallCaps]{Fontin-Regular}%SmallCaps]{Fontin-Regular}
%\setmainfont[Path = fonts/, BoldFont = LinLibertine_RB.otf, ItalicFont = LinLibertine_RI.otf, BoldItalicFont = LinLibertine_RBI.otf, SlantedFont = LinLibertine_aRL.otf, BoldSlantedFont = LinLibertine_aBL.otf, SmallCapsFont = LinLibertine_aS.otf]{LinLibertine_R.otf}
%\setmainfont[ Path = fonts/, BoldFont = GILB, BoldItalicFont = GILBI, ItalicFont = GILI, SmallCapsFont = GIL]{GIL}

%CV Sections inspired by:
%http://stefano.italians.nl/archives/26
%\titleformat{\section}{\Large\scshape\raggedright}{}{0em}{}[\titlerule]%\scshape\raggedright}{}{0em}{}[\titlerule]
%\titlespacing{\section}{0pt}{3pt}{3pt}
%Tweak a bit the top margin
%\addtolength{\voffset}{-1.3cm}

%Italian hyphenation for the word: ''corporations''
\hyphenation{im-pre-se}
\newgeometry{left=1in,right=1in,top=1in, bottom=1in}  %20140829 top 1.5-->2.5
%--------------------BEGIN DOCUMENT----------------------
\begin{document}
\thispagestyle {empty}
\pagestyle{empty}
\begin{spacing}{1.5}
%\pagestyle{empty} % non-numbered pages

%:\tiny-6pt、\scritpsize-8pt、\footnotesize-10pt、\small-11pt、

%\normalsize-12pt、\large-14pt、\Large-17pt、\LARGE-20、\huge-25pt、\Huge-25pt。

\font\fb="[cmr10]" %for use with \LaTeX command

\begin{center} {\LARGE {\textsc{Statement of Purpose}}     }  \end{center}%\textbf{{\Huge{Q}\Large{ICHEN} \Huge{S}\Large{ONG}}%}
\vspace{1em}
\begin{center} {\large {Qichen Song}     }  \end{center}
%-------------------Demarcation---------

\vspace{1.5em}
 For recent years, nanoscale heat transfer has attracted attention from researchers and engineers around the world because nano- materials have outstanding performance in energy capture, energy conversion and energy storage. I want to help to release the energy shortage.\\
\\
Like any other geeks, I have strong enthusiasm in gadgets and electronics. Specifically, I care a lot about their performance of heat dissipation because effective thermal management is becoming increasingly important for high-power electronics and portable devices. Phonon plays a key role in determining the thermal properties of semi-conductors and I want to learn related theories, methods and tools systematically to obtain an in-depth understanding of its mechanism as well as effective ways to manage phonons during my PhD studies.\\
\\
 I have an excellent academic record, especially in the course of heat transfer and thermodynamics. But most importantly, I am highly-motivated in conducting independent scientific research on nanoscale thermal transport. As a research assistant in Nano Heat Group, my research focuses on reduction of thermal conductivity of graphene for its potential thermoelectric application. Using the fortran code I wrote, I calculated the suspended graphene’s thermal conductivity by non-equilibrium molecular dynamics and compared my results with previous work by others to make sure my program is faultless. Afterwards, I changed the structure of the graphene into folded one and took the long-range atom/atom-substrate interaction into consideration. With relatively large size parameter and period boundary condition, the converged outcomes indicate the thermal conductivity of folded graphene in real size. And my results clearly show that thermal conductivity decreases with not only increasing number of folds but stronger substrate effect, which is due to the enhanced phonon scattering. I am confident that both my comprehension of mechanism of nanoscale thermal transport and programming skills were largely improved.\\
\\
 Not limited to research on the relationship between the thermal conductivity and the structure, I also actively collaborate with our group member, Dr. Meng An. The coupling between vibration modes on different directions in graphene has already been studied by others, what we do is to investigate the connection between the length and coupling. Through discussion and exchanging feedback with each other, I learned how to collaborate to overcome adversities together.\\
\\
 Admittedly, I met series of challenges in my simulation research, e.g., programming errors, the limited computing resource, but the genuine interest in nanoscale heat transfer and numerical modeling drive me forward. From these numerical studies, I acquired solid skills and rigorous attitude. Through attending group meeting every week, I catch up with the minds of pioneers. I believe these valuable experiences will greatly contribute to my further study as a graduate student in Virginia Tech.\\
\\

\end{spacing}
\end{document}
